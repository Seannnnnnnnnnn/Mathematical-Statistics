\documentclass[nocolor]{report}
% basics
\usepackage[utf8]{inputenc}
\usepackage[T1]{fontenc}
\usepackage{textcomp}
% \usepackage[dutch]{babel}
\usepackage{url}
% \usepackage{hyperref}
% \hypersetup{
%     colorlinks,
%     linkcolor={black},
%     citecolor={black},
%     urlcolor={blue!80!black}
% }
\usepackage{graphicx}
\usepackage{float}
\usepackage{booktabs}
\usepackage{enumitem}
% \usepackage{parskip}
\usepackage{emptypage}
\usepackage{subcaption}
\usepackage{multicol}
\usepackage[usenames,dvipsnames]{xcolor}

% \usepackage{cmbright}


\usepackage{amsmath, amsfonts, mathtools, amsthm, amssymb}
\usepackage{mathrsfs}
\usepackage{cancel}
\usepackage{bm}
\newcommand\N{\ensuremath{\mathbb{N}}}
\newcommand\R{\ensuremath{\mathbb{R}}}
\newcommand\Z{\ensuremath{\mathbb{Z}}}
\renewcommand\O{\ensuremath{\emptyset}}
\newcommand\Q{\ensuremath{\mathbb{Q}}}
\newcommand\C{\ensuremath{\mathbb{C}}}
\DeclareMathOperator{\sgn}{sgn}
\usepackage{systeme}
\let\svlim\lim\def\lim{\svlim\limits}
\let\implies\Rightarrow
\let\impliedby\Leftarrow
\let\iff\Leftrightarrow
\let\epsilon\varepsilon
\usepackage{stmaryrd} % for \lightning
\newcommand\contra{\scalebox{1.1}{$\lightning$}}
% \let\phi\varphi





% correct
\definecolor{correct}{HTML}{009900}
\newcommand\correct[2]{\ensuremath{\:}{\color{red}{#1}}\ensuremath{\to }{\color{correct}{#2}}\ensuremath{\:}}
\newcommand\green[1]{{\color{correct}{#1}}}



% horizontal rule
\newcommand\hr{
    \noindent\rule[0.5ex]{\linewidth}{0.5pt}
}


% hide parts
\newcommand\hide[1]{}



% si unitx
\usepackage{siunitx}
\sisetup{locale = FR}
% \renewcommand\vec[1]{\mathbf{#1}}
\newcommand\mat[1]{\mathbf{#1}}


% tikz
\usepackage{tikz}
\usepackage{tikz-cd}
\usetikzlibrary{intersections, angles, quotes, calc, positioning}
\usetikzlibrary{arrows.meta}
\usepackage{pgfplots}
\pgfplotsset{compat=1.13}


\tikzset{
    force/.style={thick, {Circle[length=2pt]}-stealth, shorten <=-1pt}
}

% theorems
\makeatother
\usepackage{thmtools}
\usepackage[framemethod=TikZ]{mdframed}
\mdfsetup{skipabove=1em,skipbelow=0em}


\theoremstyle{definition}

\declaretheoremstyle[
    headfont=\bfseries\sffamily\color{ForestGreen!70!black}, bodyfont=\normalfont,
    mdframed={
        linewidth=2pt,
        rightline=false, topline=false, bottomline=false,
        linecolor=ForestGreen, backgroundcolor=ForestGreen!5,
    }
]{thmgreenbox}

\declaretheoremstyle[
    headfont=\bfseries\sffamily\color{NavyBlue!70!black}, bodyfont=\normalfont,
    mdframed={
        linewidth=2pt,
        rightline=false, topline=false, bottomline=false,
        linecolor=NavyBlue, backgroundcolor=NavyBlue!5,
    }
]{thmbluebox}

\declaretheoremstyle[
    headfont=\bfseries\sffamily\color{NavyBlue!70!black}, bodyfont=\normalfont,
    mdframed={
        linewidth=2pt,
        rightline=false, topline=false, bottomline=false,
        linecolor=NavyBlue
    }
]{thmblueline}

\declaretheoremstyle[
    headfont=\bfseries\sffamily\color{RawSienna!70!black}, bodyfont=\normalfont,
    mdframed={
        linewidth=2pt,
        rightline=false, topline=false, bottomline=false,
        linecolor=RawSienna, backgroundcolor=RawSienna!5,
    }
]{thmredbox}

\declaretheoremstyle[
    headfont=\bfseries\sffamily\color{RawSienna!70!black}, bodyfont=\normalfont,
    numbered=no,
    mdframed={
        linewidth=2pt,
        rightline=false, topline=false, bottomline=false,
        linecolor=RawSienna, backgroundcolor=RawSienna!1,
    },
    qed=\qedsymbol
]{thmproofbox}

\declaretheoremstyle[
    headfont=\bfseries\sffamily\color{RawSienna!70!black}, bodyfont=\normalfont,
    numbered=no,
    mdframed={
        linewidth=2pt,
        rightline=false, topline=false, bottomline=false,
        linecolor=RawSienna, backgroundcolor=RawSienna!1,
    },
    qed=\qedsymbol
]{thmproofbox}

\declaretheoremstyle[
    headfont=\bfseries\sffamily\color{NavyBlue!70!black}, bodyfont=\normalfont,
    numbered=no,
    mdframed={
        linewidth=2pt,
        rightline=false, topline=false, bottomline=false,
        linecolor=NavyBlue, backgroundcolor=NavyBlue!1,
    },
]{thmexplanationbox}



% \declaretheoremstyle[headfont=\bfseries\sffamily, bodyfont=\normalfont, mdframed={ nobreak } ]{thmgreenbox}
% \declaretheoremstyle[headfont=\bfseries\sffamily, bodyfont=\normalfont, mdframed={ nobreak } ]{thmredbox}
% \declaretheoremstyle[headfont=\bfseries\sffamily, bodyfont=\normalfont]{thmbluebox}
% \declaretheoremstyle[headfont=\bfseries\sffamily, bodyfont=\normalfont]{thmblueline}
% \declaretheoremstyle[headfont=\bfseries\sffamily, bodyfont=\normalfont, numbered=no, mdframed={ rightline=false, topline=false, bottomline=false, }, qed=\qedsymbol ]{thmproofbox}
% \declaretheoremstyle[headfont=\bfseries\sffamily, bodyfont=\normalfont, numbered=no, mdframed={ nobreak, rightline=false, topline=false, bottomline=false } ]{thmexplanationbox}

\declaretheorem[style=thmgreenbox, name=Definition]{definition}
\declaretheorem[style=thmbluebox, numbered=no, name=Example]{eg}
\declaretheorem[style=thmredbox, name=Proposition]{prop}
\declaretheorem[style=thmredbox, numbered=no, name=Exercise]{ex}
% \declaretheorem[style=thmproofbox, name=Solution]{soln}
\declaretheorem[style=thmredbox, name=Theorem]{theorem}
\declaretheorem[style=thmredbox, name=Lemma]{lemma}
\declaretheorem[style=thmredbox, numbered=no, name=Corollary]{corollary}

\declaretheorem[style=thmproofbox, name=Proof]{replacementproof}
\renewenvironment{proof}[1][\proofname]{\vspace{-10pt}\begin{replacementproof}}{\end{replacementproof}}



\declaretheorem[style=thmexplanationbox, name=Proof]{tmpexplanation}
\newenvironment{explanation}[1][]{\vspace{-10pt}\begin{tmpexplanation}}{\end{tmpexplanation}}

\declaretheorem[style=thmblueline, numbered=no, name=Remark]{remark}
\declaretheorem[style=thmblueline, numbered=no, name=Note]{note}

\newtheorem*{uovt}{UOVT}
\newtheorem*{notation}{Notation}
\newtheorem*{previouslyseen}{As previously seen}
\newtheorem*{problem}{Problem}
\newtheorem*{observe}{Observe}
\newtheorem*{property}{Property}
\newtheorem*{intuition}{Intuition}


\usepackage{etoolbox}
\AtEndEnvironment{vb}{\null\hfill$\diamond$}%
\AtEndEnvironment{intermezzo}{\null\hfill$\diamond$}%
% \AtEndEnvironment{opmerking}{\null\hfill$\diamond$}%

% http://tex.stackexchange.com/questions/22119/how-can-i-change-the-spacing-before-theorems-with-amsthm
\makeatletter
% \def\thm@space@setup{%
%   \thm@preskip=\parskip \thm@postskip=0pt
% }

\newcommand{\oefening}[1]{%
    \def\@oefening{#1}%
    \subsection*{Oefening #1}
}

\newcommand{\suboefening}[1]{%
    \subsubsection*{Oefening \@oefening.#1}
}

\newcommand{\exercise}[1]{%
    \def\@exercise{#1}%
    \subsection*{Exercise #1}
}

\newcommand{\subexercise}[1]{%
    \subsubsection*{Exercise \@exercise.#1}
}


\usepackage{xifthen}

\def\testdateparts#1{\dateparts#1\relax}
\def\dateparts#1 #2 #3 #4 #5\relax{
    \marginpar{\small\textsf{\mbox{#1 #2 #3 #5}}}
}

\def\@lesson{}%
\newcommand{\lesson}[3]{
    \ifthenelse{\isempty{#3}}{%
        \def\@lesson{Lecture #1}%
    }{%
        \def\@lesson{Lecture #1: #3}%
    }%
    \subsection*{\@lesson}
    \testdateparts{#2}
}

% \renewcommand\date[1]{\marginpar{#1}}


% fancy headers
\usepackage{fancyhdr}
\pagestyle{fancy}

% \fancyhead[LE,RO]{Gilles Castel}
\fancyhead[RO,LE]{\@lesson}
\fancyhead[RE,LO]{}
\fancyfoot[LE,RO]{\thepage}
\fancyfoot[C]{\leftmark}

\makeatother




% notes
\usepackage{todonotes}
\usepackage{tcolorbox}

\tcbuselibrary{breakable}
\newenvironment{verbetering}{\begin{tcolorbox}[
    arc=0mm,
    colback=white,
    colframe=green!60!black,
    title=Opmerking,
    fonttitle=\sffamily,
    breakable
]}{\end{tcolorbox}}

\newenvironment{noot}[1]{\begin{tcolorbox}[
    arc=0mm,
    colback=white,
    colframe=white!60!black,
    title=#1,
    fonttitle=\sffamily,
    breakable
]}{\end{tcolorbox}}




% figure support
\usepackage{import}
\usepackage{xifthen}
\pdfminorversion=7
\usepackage{pdfpages}
\usepackage{transparent}
\newcommand{\incfig}[1]{%
    \def\svgwidth{\columnwidth}
    \import{./figures/}{#1.pdf_tex}
}

% %http://tex.stackexchange.com/questions/76273/multiple-pdfs-with-page-group-included-in-a-single-page-warning
\pdfsuppresswarningpagegroup=1


\author{Sean Conlon}


\begin{document}
    \newpage
    \subsection*{MAST90012 Measure Theory. Assignment 1.}
\begin{center}
    Sean Conlon, 1298668 \\
    sconlon@student.unimelb.edu.au
\end{center}


% problem 1 
\begin{ex}[Question 1.
] Let $F$ denote the \textit{generalised Cantor set}, where the intervals removed at the $k$-th step have length $3^{-k}\alpha$ for $0<\alpha<1$. Show that $F$ is closed, $F^c$ is dense in $[0,1]$ and $m(F) = 1-\alpha$
\end{ex}
\begin{proof}
We first note that $F$ can be written as 
$$F = [0,1]\backslash \bigcup_{n=1}^{\infty}\bigcup_{i=1}^{2^n -1} I_{nm}$$
where $I_{nm}$ are the $m=1,\dots, 2^{n-1}$ closed intervals of length $\alpha3^{-n}$. We now prove each of the following statements. \\
\\
\textit{F is closed}. $F$ is formed through a process of a removing closed sub-intervals from the interval $[0,1]$. As closed intervals remain closed under union and intersection, we conclude that $F$ must also be closed. \\
\\
\textit{$F^c$ is dense in $[0,1]$}. Let $x,y\in[0,1]$ such that $x<y$ and $y-x > \alpha/3^k$ Thus, no interval of length $\alpha/3^k$ contains the open interval $(x,y)$ as a subset. Therefore $(x,y)\not\subset F_k \implies (x,y)\not\subset F$. Therefore, there exists a $c\in(x,y)$ such that $c\notin F$. We can then conclude that $c\in F^c$ and thus $F^c$ is dense in $[0,1]$ \\
\\
\textit{F is measurable and has lebegsgue measure $m(F) = 1-\alpha$}. $F^c$  is comprised of the countable union of closed, measurable subintervals of $\mathbb{R}$ and is therefore measurable. Thus $F$ is itself measurable. To determine it's Lebesgue measure, we note that $F$ can be written as 
$$F = [0,1]\backslash \bigcup_{n=1}^{\infty}\bigcup_{i=1}^{2^n -1} I_{nm}$$
Where $I_{nm}$ are the pairwise disjoint intervals of length $|I_{nm}| = \alpha 3^{-n}$ It then follows 
\begin{align*}
    m\left([0,1]\backslash \bigcup_{n=1}^{\infty}\bigcup_{i=1}^{2^n -1} I_{nm} \right) &= m([0,1]) - m\left( \bigcup_{n=1}^{\infty}\bigcup_{i=1}^{2^n -1} I_{nm}\right) \\
    &= 1 - \sum_{n=1}^{\infty}\sum_{m=1}^{2^n -1} m(I_{nm}) \\
    &= 1 - \sum_{n=1}^{\infty}\sum_{m=1}^{2^n -1} |I_{nm}| \\
    &= 1 - \sum_{n=1}^{\infty} \alpha \frac{2^{n-1}}{3^n} \\
    &= 1 - \alpha
\end{align*}


\end{proof}

% problem 2
\newpage
\begin{ex}[Question 2.]
Prove that if $E\subset\mathbb{R}^d$ is measurable then $\delta E$ is measurable. Moreover, show that $$m(\delta E) = \delta m(E)$$
\end{ex}
\begin{proof} 
For notational convenience, we define $\Delta := \delta_1 \time \dots \delta_d$ and
first note that if $Q$ is a (closed) cube such that $Q = [a_1, b_1] \times \dots \times [a_d, b_d]$ then $\delta Q = [\delta_1a_1, \delta_1 b_1] \times \dots \times [\delta_d a_d, \delta_d b_d]$ with Lebesgue measure 
\begin{align*}
    m(\delta Q) &= \prod_{i=1}^{d} \delta_i [a_i - b_i] \\
    &= \Delta |Q|
\end{align*}
We now proceed with the proof. Let $E$ be a measurable set, so that there exists an open set $U$ such that $E\subset U$ and 
$$m^*(U\backslash E) \leq \frac{\varepsilon}{\Delta}$$
Clearly, we  have $E\subset U \implies \delta E\subset \delta U$ with $\delta U$ an open set. It then follows from the definition of the outer measure
\begin{align*}
    m^*(\delta U \backslash \delta E) &= \inf \left\{ \sum_{j=1}^{\infty} m^*(\delta Q_j)  \hspace{3mm} \bigg| \hspace{3mm} U\backslash E \subset\bigcup_{j=1}^{\infty} Q_j \right\} \\
    &= \inf \left\{ \Delta\sum_{j=1}^{\infty} |Q_j|  \hspace{3mm} \bigg| \hspace{3mm} U\backslash E \subset\bigcup_{j=1}^{\infty} Q_j \right\} \\
    &= \Delta m(U\backslash E) \\
    &\leq \Delta \frac{\varepsilon}{\Delta} = \varepsilon
\end{align*}
And thus $\delta E$ is measurable. As we have somewhat shown above, we can determine the Lebesgue measure of a measurable set $\delta E$ by 
\begin{align*}
    m(\delta E) &= \inf \left\{ \sum_{j=1}^{\infty} m^*(\delta Q_j)  \hspace{3mm} \bigg| \hspace{3mm} U\backslash E \subset\bigcup_{j=1}^{\infty} Q_j \right\} \\
    &= \inf \left\{ \Delta\sum_{j=1}^{\infty} |Q_j|  \hspace{3mm} \bigg| \hspace{3mm} U\backslash E \subset\bigcup_{j=1}^{\infty} Q_j \right\} \\
    &= \Delta m^*(E)
\end{align*}
\end{proof}


% problem 3
\newpage
\begin{ex}[Question 3. a] For each subset $E$ of $\mathbb{R}$ the \textbf{outer Jordan content} $J^*(E)$ of $E$ is defined by
$$J^*(E) = \inf\sum_{j=1}^{N}|I_j|$$
where the infimim is taken over every \textit{finite} covering $E\subset \cup_{j=1}^{N}I_j$ by closed intervals $I_j$. \\
\\
Prove that $J^*(E) = J^*(\Bar{E})$ for every set $E$, where $\Bar{E}$ denotes the closure of $E$.
\end{ex}
\begin{proof} We first show that $J^*(E) \leq J^*(\Bar{E})$.\\ 
\\
Let $E\subset\mathbb{R}$ be an open set, and $\{I_j\}_{j=1}^{N}$ a covering of its closure, $\subset\Bar{E}$. Since $E\subset\Bar{E}$, $\{I_j\}_{j=1}^{\infty}$ is a covering of $E$ also, thus $J^*(E)\leq J^*(\Bar{E})$. \\
\\
We now show that $J^*(E) \geq J^*(\Bar{E})$. Let $I = \{I_j\}_{j=1}^{N}$ be a covering of $E$. By contradiction, suppose that $\bar{I}$ does not cover $\Bar{E}$. Then, there must exist $x\in E$ such that $x\nin \Bar{I}$. By defintion, $\Bar{E} = E \cup E^*$ where $E^*$ is the set of limit points of $E$. Thus, for all $\varepsilon>0$ $(x-\varepsilon, x+\varepsilon)\cap E = \emptyset$ but $E\subset\Bar{I}$ which therefore implies $(x-\varepsilon, x+\varepsilon)\cap \Bar{I} \neq \emptyset$ and therefore $x$ is a limit point of $\Bar{I}$. \\
\\
However, as $\bar{I}$ is a finite union of closed intervals, $\bar{I}$ is closed and contains all of its limit points. Thus, $x\in\bar{I}$ is a contradcation. Therefore $E\subset I$ then $\bar{E}\subset\bar{I}$ which implies $J^*(E)\geq J^*(\bar{E})$.
\end{proof}
\begin{ex}[Question 3. b]
    Demonstrate that there exists a subset $E\subset[0,1]$ such that $J^*(E)=1$ and $m^*(E)=0$
\end{ex}
\begin{proof}
     An often useful set to consider in analysis is $I = \mathbb{Q}\cap[0,1]$. By the countability of $\mathbb{Q}$, $I$ is clearly a countable, proper subset of $[0,1]$. Using the fact that $\mathbb{Q}$ is dense in $\mathbb{R}$ we have 
$$J^*(I) = J^*(\bar{I}) = J^*([0,1]) = 1$$
Moreover, since $I$ is the set of rationals within $[0,1]$, we can express $I$ as  $I = \cup_{q\in \mathbb{Q}: 0\leq q\leq 1}\{q\}$ Using the subadditivity of the Lebesgue outer measure we have 
$$0\leq m^*(I) \leq \sum_{q\in I} m^*(\{q\}) = 0$$
\end{proof}



   






% problem 4
\newpage
\begin{ex}[Question 4.] Suppose $E$ is a measurable subset of $\mathbb{R}^d$ such that $m(E)<\infty$, $E_1, E_2$ are subsets of $\mathbb{R}^d$ such that 
$$E = E_1 \cup E_2 \hspace{5mm} E_1\cap E_2 = \emptyset$$
Show that if $m(E) = m^*(E_1) + m^*(E_2)$ then $E_1, E_2$ are measurable sets. In particular, when $E\subset Q$ for some closed cube, $E$ is measurable if and only if $m(Q) = m^*(E) + m^*(Q\backslash E)$
\end{ex}
\begin{proof}
To prove the first statement, let $\varepsilon>0$ and $U_i$ an open subset of $\mathbb{R}^d$ such that for $i=1,2$ 
$$m^*(E_i)\leq m^*(U_i) < m^*(E_i) + \frac{\varepsilon}{2}$$
This implies $E \subset U_1 \cup U_2$ and by subadditivity 
\begin{align*}
    m^*(E) &\leq m^*(U_1 \cup U_2) \\
    & = m^*(U_1) + m^*(U_2) - m^*(U_1 \cap U_2)
\end{align*}
Note that we can rearrange the above so that 
\begin{align*}
    m^*(U_1 \cap U_2) &\leq m^*(U_1) + m^*(U_2) - m^*(E) \\ 
    &= m^*(U_1) + m^*(U_2) - m^*(E_1) - m^*(E_2) \\
    &< \varepsilon
\end{align*}
Finally, we note that $U_i \backslash E_i \subset (U_1 \cap U_2) \cup ((U_1 \cup U_2)\backslash E)$ and it follows that 
\begin{align*}
    m^*(U_i \backslash E_i) &\leq m^*(U_1 \cap U_2) + m^*(U_1 \cup U_2) - m^*(E) \\
    &\leq \varepsilon + m(U_1) - m(E_1) + m(U_2) - m(E_2) \\
    &= 2\varepsilon
\end{align*}
As $\varepsilon>0$ is arbitrary, we have shown that $E_1, E_2$ are measurable.
\end{proof}
\begin{proof}
    We now prove the latter statement: If $E\subset Q$ for some closed cube, $E$ is measurable if and only if $m(Q) = m^*(E) + m^*(Q\backslash E)$. \\
    \\
    $(\Rightarrow)$. \\
    Let $E$ be measurable, and $Q$ a cube covering $E$. By subadditiviy we have 
    $$m^*(Q) \leq m^*(E) + m^*(Q\backslash E)$$
    To show the inverse inequality, we note that since $E$ is measurable, $\exists$ an open set $U$ such that $m^*(U \backslash E) \leq \varepsilon$. In particular, we choose $\varepsilon = m^*(U) - m^*(E) > 0$ as $E\subset U$ so that 
    $$m^*(U\backslash E) + m^*(E) \leq m^*(U)$$
    We now let $Q$ be a covering of $U$ such that $E\subset U\subseteq Q$ and by monotonicity we have 
    \begin{align*}
        m^*(U\backslash E) + m^*(E) \leq m^*(Q \backslash E) +m^*(E) \leq m^*(Q)
    \end{align*}
    Thus we have proven the claim. \\
    \\
    $(\Leftarrow)$. \\
    Let $E\subset Q$ such that $m(Q) = m^*(E) + m^*(Q\backslash E)$, we aim to show that $E$ is measurable. \\
    \\
    Let $\varepsilon>0$ and $U$ be an open covering of $Q$ such that $E\subset Q\subseteq U$ such that 
    \begin{align*}
        m^*(U)\leq m^*(E) + \varepsilon
    \end{align*}
    It now follows that 
    \begin{align*}
        m^*(Q) &= m^*(E) + m^*(Q\backslash E) \\
               &\leq m^*(E) + m^*(U\backslash E) \\
               &= m^*(U) 
    \end{align*}
    Combining the above with $(1)$ we find that 
    $$m^*(U\backslash E) \leq \varepsilon$$
    and therefore conclude that $E$ is measurable. \\
\end{proof}



\end{document}